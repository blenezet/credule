\documentclass[a4paper]{book}
\usepackage[times,inconsolata,hyper]{Rd}
\usepackage{makeidx}
\usepackage[utf8,latin1]{inputenc}
% \usepackage{graphicx} % @USE GRAPHICX@
\makeindex{}
\begin{document}
\chapter*{}
\begin{center}
{\textbf{\huge \R{} documentation}} \par\bigskip{{\Large of \file{credule\man\bootstrapCDS.Rd}}}
\par\bigskip{\large \today}
\end{center}
\inputencoding{utf8}
\HeaderA{bootstrapCDS}{Bootstrap a Credit Curve}{bootstrapCDS}
\keyword{CDS, Credit Default Swap, pricing}{bootstrapCDS}
%
\begin{Description}\relax
A function that bootstrap a credit curve from a set of Credit Default Swap spreads givent for various maturity.
\end{Description}
%
\begin{Usage}
\begin{verbatim}
bootstrapCDS(yieldcurveTenor,
			 yieldcurveRate,
			 cdsTenors,
			 cdsSpreads,
			 recoveryRate,
			 numberPremiumPerYear = c(4,2,1,12),
			 numberDefaultIntervalPerYear = 12,
			 accruedPremium = c(TRUE,FALSE)) 
\end{verbatim}
\end{Usage}
%
\begin{Arguments}
\begin{ldescription}
\item[\code{yieldcurveTenor}] 
A double vector. Each value represents a tenor of the yield curve expressed in year (e.g. 1.0 for 1Y, 0.5 for 6M)

\item[\code{yieldcurveRate}] 
A double vector. Each value represents the discount rate (continuously compounded) for a partical tenor (e.g. 0.005 means 0.5\%, 0.02 means 2

\item[\code{cdsTenors}] 
A double vector. Each value represents the maturity expressed in year of a Credit Default Swap which we want to price (e.g 5.0 means 5Y)

\item[\code{cdsSpreads}] 
A double vector. Each value represents the CDS spread (expressed in decimal, e.g. 0.0050 represent 0.5\% or 50 bp) for a given maturity

\item[\code{recoveryRate}] 
A double. It represents the Recovery Rate in case of default (e.g 0.40 means 40\% recovery which is a standard value for Senior Unsecured debt)

\item[\code{numberPremiumPerYear}] 
An Integer. It represents the number of premiums paid per year. CDS premiums paid quaterly (i.e. numberPremiumPerYear=4) and sometimes semi-annually (i.e. numberPremiumPerYear=2)
	
\item[\code{numberDefaultIntervalPerYear}] 
An Integer. It represents the number of timesteps used to perform the numerical integral required while computing the default leg value. It is shown that a monthly discretisation usually gives a good approximation (Ref. Valuation of Credit Default Swaps, Dominic O Kane and Stuart Turnbull)
	
\item[\code{accruedPremium}] 
A boolean. If set to TRUE, the accrued premium will be taken into account in the calculation of the premium leg value.
	
\end{ldescription}
\end{Arguments}
%
\begin{Value}
Returns a Dataframe with 3 columns: tenor, survprob and hazrate. The tenor column contains the tenor value given in parameter cdsTenors, the survprob column gives the survival probability (in decimal) for each tenor (e.g. 0.98 menas 98\%) and the hazrate column gives the non-cumulative hazard rate (intensity of the poisson process) for each tenor (e.g. 0.01 means 1\% hazard rate).
\end{Value}
%
\begin{Author}\relax
Bertrand Le Nezet
\end{Author}
%
\begin{Examples}
\begin{ExampleCode}
library(credule)

yieldcurveTenor = c(1,2,3,4,5,7)
yieldcurveRate = c(0.0050,0.0070,0.0080,0.0100, 0.0120,0.0150)
cdsTenors = c(1,3,5,7)
cdsSpreads = c(0.0050,0.0070,0.0090,0.0110)
premiumFrequency = 4
defaultFrequency = 12
accruedPremium = TRUE
RR = 0.40

bootstrapCDS(yieldcurveTenor,
             yieldcurveRate,
             cdsTenors,
             cdsSpreads,
             RR,
             premiumFrequency,
             defaultFrequency,
             accruedPremium)
\end{ExampleCode}
\end{Examples}
\printindex{}
\end{document}
